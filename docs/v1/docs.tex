\PassOptionsToPackage{unicode=true}{hyperref} % options for packages loaded elsewhere
\PassOptionsToPackage{hyphens}{url}
%
\documentclass[]{article}
\usepackage{lmodern}
\usepackage{amssymb,amsmath}
\usepackage{ifxetex,ifluatex}
\usepackage{fixltx2e} % provides \textsubscript
\ifnum 0\ifxetex 1\fi\ifluatex 1\fi=0 % if pdftex
  \usepackage[T1]{fontenc}
  \usepackage[utf8]{inputenc}
  \usepackage{textcomp} % provides euro and other symbols
\else % if luatex or xelatex
  \usepackage{unicode-math}
  \defaultfontfeatures{Ligatures=TeX,Scale=MatchLowercase}
\fi
% use upquote if available, for straight quotes in verbatim environments
\IfFileExists{upquote.sty}{\usepackage{upquote}}{}
% use microtype if available
\IfFileExists{microtype.sty}{%
\usepackage[]{microtype}
\UseMicrotypeSet[protrusion]{basicmath} % disable protrusion for tt fonts
}{}
\IfFileExists{parskip.sty}{%
\usepackage{parskip}
}{% else
\setlength{\parindent}{0pt}
\setlength{\parskip}{6pt plus 2pt minus 1pt}
}
\usepackage{hyperref}
\hypersetup{
            pdfborder={0 0 0},
            breaklinks=true}
\urlstyle{same}  % don't use monospace font for urls
\usepackage{longtable,booktabs}
% Fix footnotes in tables (requires footnote package)
\IfFileExists{footnote.sty}{\usepackage{footnote}\makesavenoteenv{longtable}}{}
\setlength{\emergencystretch}{3em}  % prevent overfull lines
\providecommand{\tightlist}{%
  \setlength{\itemsep}{0pt}\setlength{\parskip}{0pt}}
\setcounter{secnumdepth}{0}
% Redefines (sub)paragraphs to behave more like sections
\ifx\paragraph\undefined\else
\let\oldparagraph\paragraph
\renewcommand{\paragraph}[1]{\oldparagraph{#1}\mbox{}}
\fi
\ifx\subparagraph\undefined\else
\let\oldsubparagraph\subparagraph
\renewcommand{\subparagraph}[1]{\oldsubparagraph{#1}\mbox{}}
\fi

% set default figure placement to htbp
\makeatletter
\def\fps@figure{htbp}
\makeatother


\date{}

\begin{document}

\hypertarget{gemsec-neuropeptide-docs}{%
\section{GEMSEC Neuropeptide Docs}\label{gemsec-neuropeptide-docs}}

\textbf{January 28, 2019}

\emph{Chris Pecunies, Aaron, Savvy Gupta}

\emph{Led by: Jacob Rodriguez}

\emph{PI: Mehmet Sarikaya}

\hypertarget{abstract}{%
\subsection{Abstract}\label{abstract}}

The purpose of this documentation is to serve as a reference for the
motivation, background, methods, and results of the GEMSEC project to
discover the method of signal transduction in human neuropeptides, to
ultimately discover candidate peptides with binding affinity to
graphene, as well as to illuminate their mechanism of binding.

\hypertarget{introduction}{%
\subsection{Introduction}\label{introduction}}

The overarching aim of this project is to uncover the mechanism of
signal transduction in neuropeptides which are experimentally determined
to bind to graphene. To determine candidate peptides to be experimented
on, we have utilized three principal methods: cross-sequence entropy,
PAM30 sequence similarity, and RRM\footnote{dfsdfsfs} relative to two
known binders, \emph{GrBP5}\footnote{fdsfs} and \emph{M6}\footnote{dfsdfsf}

\hypertarget{background}{%
\subsection{Background}\label{background}}

While there is extensive literature surrounding the subject of
peptide-solid binding and its most important determinants, we have
focused on two primary methods: sequence domain functionality clustering
and the \emph{Resonant Recognition Model} proposed by Irena Cosic
\footnote{dfsdfs}.

\hypertarget{methods}{%
\subsection{Methods}\label{methods}}

\hypertarget{rrm}{%
\paragraph{RRM}\label{rrm}}

The \emph{Resonant Recognition Model}, or RRM, is a method proposed by
Irena Cosic for the analysis of peptides and proteins. The model assumes
electron-ion interaction potential (EIIP, hereafter) along the backbone
of amino acid chains to be the most significant deteriminant in a
peptide or protein's biochemical phenomenological features. For each
amino acid, a unique scalar EIIP value can be determined, which allows a
amino acid sequence to be converted to a scalar sequence of EIIP values
in its most simple form.

\hypertarget{cross-entropy}{%
\paragraph{Cross entropy}\label{cross-entropy}}

We determined the cross entropy for both the pure amino acid sequences
as well as the functional domain numerical sequences, which can be found
in the appendix (and which will be generated in the main program).

\hypertarget{pam30-sequence-similarity}{%
\paragraph{PAM30 Sequence similarity}\label{pam30-sequence-similarity}}

The PAM (point accepted mutation) matrices, initially developed by
Margaret Dayhoff\footnote{dsfsdfs} are commonly used in measuring the
similarity of two peptide or protein sequences. The PAM30 matrix,
specifically, is a sequence alignment matrix that allows 30 point
accepted per 100 amino acids. The PAM30 matrix is a ``shallow'' sequence
alignment matrix, in that it is more appropriate in determining
alignment for more ``similar'' sequences.

\hypertarget{results}{%
\subsection{Results}\label{results}}

\hypertarget{candidate-peptides}{%
\paragraph{Candidate peptides}\label{candidate-peptides}}

The five similarity metrics generated for each neuropeptide served as a
feature set which could be weighted and used to train a regression
model, but without experimental data, the relative weights for each
feature were chosen by {[}METHOD{]}.

After forwarding the candidate peptides for experimentation, we utilized
statistical clustering and signal processing methods to predict
determinant sequence domains in graphene binding which could then be
used with the experimental data once acquired.

\hypertarget{experimental-results}{%
\paragraph{Experimental results}\label{experimental-results}}

After generating the list of {[}NUMBER{]} candidate peptides and
receiving binding affinity metrics, we were then able to perform several
statistical methods to ascertain the most influential factors in peptide
graphene binding.

First, we trained a simple linear regression model, using the gathered
experimental results as training data, and {[}\ldots{}{]}

\hypertarget{signal-transduction}{%
\paragraph{Signal transduction}\label{signal-transduction}}

From results gathered from model training, statistical clustering, and
experimental cross-validation, we were able to identify several sequence
patterns and corresponding functions which may prove significant.
{[}\ldots{}{]}

\hypertarget{discussion}{%
\subsection{Discussion}\label{discussion}}

\hypertarget{computational-results}{%
\paragraph{Computational results}\label{computational-results}}

Our program, using all aforementioned computational methods, generated
two ``similarity tables'' for 934,235 {[}@TODO: revise this{]}
neuropeptides: one for GrBP5, and one for the wild-type peptide M6. This
generated a table of the schema (example data -- replace with top 10-20
candidate peptides):

\textbf{\emph{Table 1:}} \emph{Output schema of .csv similarity tables}

\begin{longtable}[]{@{}lllllll@{}}
\toprule
\textbf{seq} & numseq & rrmsn & rrmcorr & pam30 & numentr &
aaentr\tabularnewline
\midrule
\endhead
IMVSTED & 1132335 & 73 & 88 & 23 & 45 & 55\tabularnewline
GTTYUEI & 3224121 & 11 & 5 & 6 & 14 & 13\tabularnewline
\bottomrule
\end{longtable}

Where the columns correspond as follows:

\begin{itemize}
\tightlist
\item
  \textbf{seq}: Original amino acid sequence of the neuropeptide
\item
  \textbf{numseq}: The original amino acid sequence converted to numbers
  corresponding to their biological function as determined by the
  biological function key-value table\footnote{See table 1 in appendix}
\item
  \textbf{rrmsn}: The signal-to-noise ratio of the cross-signal of the
  EIIP frequency spectum calculated for a given neuropeptide and the
  sequence of comparison, normalized in the domain {[}0, 100{]}.
\item
  \textbf{rrmcor}: The Pearson correlation coefficient between the EIIP
  frequency spectrum calculated for a given neuropeptide and the
  sequence of comparison, mapped to the domain {[}0, 100{]}
\item
  \textbf{pam30}: The PAM30 similarity score (formula described in
  methods section); a measure of inverse distance using the PAM30
  matrix, mapped to {[}0, 100{]}
\item
  \textbf{numentr}: The cross-entropy of the numerical sequence
  represencted by \texttt{numseq}
\item
  \textbf{aaentr}: The cross-entropy of the amino acid sequence
\end{itemize}

From this data, we determined that {[}\ldots{}{]}

\hypertarget{experimental-results-1}{%
\paragraph{Experimental results}\label{experimental-results-1}}

{[}to be completed after experimentaiton{]}

\hypertarget{clustering-and-signal-analysis}{%
\paragraph{Clustering and signal
analysis}\label{clustering-and-signal-analysis}}

After gathering the experimental results, we were then able to apply
supervised statistical learning techniques to the full set of
neuropeptides, as well as signal processing techniques. We first
{[}\ldots{}{]}

\hypertarget{how-to-run}{%
\subsection{How to run}\label{how-to-run}}

\textbf{Prerequisites}: A computer running Windows/Mac OSX/Linux with
\href{https://www.python.org/}{Python} (3+) installed, and the Python
libraries \href{https://pandas.pydata.org/}{pandas},
\href{https://numpy.org/}{NumPy}, and
\href{https://scikit-learn.org/stable/}{scikit-learn}. For
visualizations, the library \href{https://matplotlib.org/}{matplotlib}
should be installed. If these libraries are not already installed,
instructions for their installation will be listed below.

\begin{enumerate}
\def\labelenumi{\arabic{enumi}.}
\item
  Download the \href{http://google.com}{.zip containing all .py files
  and sample data set} \textbf{{[}!FIX{]}} and extract to preferred
  location.
\item
  Open a terminal and navigate to the directory containing the extracted
  files

  \begin{enumerate}
  \def\labelenumii{\arabic{enumii}.}
  \item
    On Windows, press the Windows key and type ``cmd'', and the press
    enter. \texttt{dir} lists all files and folders in the working
    directory, while \texttt{cd\ dirName} changes the working directory
    to the specified folder (in this case, \texttt{dirname}). To move up
    the directory hierarchy, type \texttt{cd\ ../}
  \item
    On Mac OSX, enter spotlight search with Command + Spacebar and type
    ``Terminal'', then hit enter. Instructions are the same as for
    windows, but replace \texttt{dir} with \texttt{ls}.
  \item
    On Linux, a terminal is likely easily accessible. Commands are the
    same as for Mac OSX.
  \end{enumerate}
\item
  When in the directory containing the extracted files, type the
  following command to generate the ``similarity tables'' for the
  example dataset: \texttt{python\ main.py\ example\_data.csv}

  \begin{itemize}
  \item
    This can be run with any .csv sequence file, but it must follow the
    same schema as the example\_data.csv table provided, and the
    sequences must all be of the same length relative to each other as
    well as to the known binder(s) (non length-matching input sequences
    will be ignored in output)
  \item
    \emph{(To implement)} The script accepts as a second argument a
    sequence (or list of sequences), each of which will generate its own
    similarity table for the input .csv.

    \begin{itemize}
    \item
      If no argument is given (as above), it will generate two .csv
      similarity tables, one for GrBP5 and one for M6. To specify only
      one .csv output similarity table (for GrBP5), run
      \texttt{python\ main.py\ example\_data.csv\ grbp5}.
    \item
      This can be done for an arbitrary number of different sequences,
      for example:
      \texttt{python\ main.py\ example\_data.csv\ IVTSSY\ UVGEASTT\ EEVTUSGMII}
      will output three .csv tables for peptides in
      \texttt{example\_data.csv} of lengths corresponding to each
      sequence specified by the user.
    \item
      Finally, the second argument can itself be a .csv of sequences,
      following the same schema as the first input .csv. In this way,
      for a .csv entered as a second argument with 10 rows of sequences
      will generate 10 separate .csv tables.
    \end{itemize}
  \end{itemize}
\item
  Similarity tables and visualizations will be generated in a folder
  titled ``output'' located in the same directory as the
  \texttt{main.py} file.
\end{enumerate}

\hypertarget{appendix}{%
\subsection{Appendix}\label{appendix}}

\hypertarget{table-1-aa}{%
\paragraph{Table 1: AA}\label{table-1-aa}}

\begin{longtable}[]{@{}llll@{}}
\toprule
Function & AA & Num & EIIP\tabularnewline
\midrule
\endhead
Aromatic & F & 0 &\tabularnewline
Aromatic & Y & 0 &\tabularnewline
Aromatic & W & 0 &\tabularnewline
Hydrophobic & A & 1 &\tabularnewline
Hydrophobic & V & 1 &\tabularnewline
Hydrophobic & I & 1 &\tabularnewline
Hydrophobic & L & 1 &\tabularnewline
Hydrophobic & M & 1 &\tabularnewline
Polar & S & 2 &\tabularnewline
Polar & T & 2 &\tabularnewline
Polar & N & 2 &\tabularnewline
Polar & Q & 2 &\tabularnewline
Proline & P & 3 &\tabularnewline
Glycine & G & 4 &\tabularnewline
Charge (-) & D & 5 &\tabularnewline
Charge (-) & E & 5 &\tabularnewline
Charge (+) & K & 6 &\tabularnewline
Charge (+) & H & 6 &\tabularnewline
Charge (+) & R & 6 &\tabularnewline
Excluded & C & 7 &\tabularnewline
\bottomrule
\end{longtable}

\hypertarget{bibliography}{%
\subsection{Bibliography}\label{bibliography}}

\end{document}
