\documentclass[a4paper,11pt]{article}
\usepackage[T1]{fontenc}
\usepackage[utf8]{inputenc}
\usepackage{lmodern}
\usepackage{longtable}
\usepackage{booktabs}
\usepackage{hyperref}

\urlstyle{same}

\providecommand{\tightlist}{%
  \setlength{\itemsep}{0pt}\setlength{\parskip}{0pt}}

\title{}
\author{}

\begin{document}
	
	
	

	\section{GEMSEC Neuropeptide Docs}\label{gemsec-neuropeptide-docs}
	
	\textbf{January 28, 2019}
	
	\emph{Chris Pecunies, Aaron, Savvy Gupta}
	
	\emph{Led by: Jacob Rodriguez}
	
	\emph{PI: Mehmet Sarikaya}
	
	\subsection{Abstract}\label{abstract}
	
	The purpose of this documentation is to serve as a reference for the
	motivation, background, methods, and results of the GEMSEC project to
	discover the method of signal transduction in human neuropeptides, to
	ultimately discover candidate peptides with binding affinity to
	graphene, as well as to illuminate their mechanism of binding.
	
	\subsection{Introduction}\label{introduction}
	
	The overarching aim of this project is to uncover the mechanism of
	signal transduction in neuropeptides which are experimentally determined
	to bind to graphene. To determine candidate peptides to be experimented
	on, we have utilized three principal methods: cross-sequence entropy,
	PAM30 sequence similarity, and RRM{[}\^{}1{]} relative to two known
	binders, \emph{GrBP5}{[}\^{}2{]} and \emph{M6}{[}\^{}3{]}
	
	\subsection{Background}\label{background}
	
	While there is extensive literature surrounding the subject of
	peptide-solid binding and its most important determinants, we have
	focused on two primary methods: sequence domain functionality clustering
	and the \emph{Resonant Recognition Model} proposed by Irena Cosic
	{[}\^{}4{]}.
	
	The \emph{Resonant Recognition Model}, or RRM, is a method proposed by
	Irena Cosic for the analysis of peptides and proteins. The model assumes
	electron-ion interaction potential (EIIP, hereafter) along the backbone
	of amino acid chains to be the most significant deteriminant in a
	peptide or protein's biochemical phenomenological features. For each
	amino acid, a unique scalar EIIP value can be determined, which allows a
	amino acid sequence to be converted to a scalar sequence of EIIP values
	in its most simple form.
	
	\subsection{Methods}\label{methods}
	
	\paragraph{RRM}\label{rrm}
	
	The Resonant Recognition Model in particular is a physical signal
	processing which interprets a protein or peptide's biological function
	as primarily determined by the corresponding translation numerical
	sequence of EIIP values, justified by the finding that there is
	significant correlation between the spectra of an amino acid sequence
	represented numericaly and its biological activity {[}\^{}3{]}. In our
	computational studies, we have extended the principle underlying Cosic's
	Resonant Recognition Model to another numerical amino acid mapping,
	itself a representation of the amino acid's most significant biological
	characteristic.
	
	From (where?) we used a mapping of the 20 amino acids to seven different
	categories, specifically: aromatic structure, hydrophobic function,
	polar structure, proline (itself), glycine (itself), negative charge,
	positive charge, and Cystine which is excluded (assigned to a separate
	category from all others, like proline and glycine. With the EIIP and
	biological function mapping keys, we were able to derive two unique
	numerical sequences per neuropeptide. From each of these translated
	sequences, following Irena Cosic's RRM construction, we could then
	perform a discrete Fourier transform to determine the spectrum of the
	numerical representation, which generated a characteristic peak in most
	cases which can be considered to be representative of an amino acid
	sequence's biological function.
	
	Because our experiment is principally concerned with the relative
	similarity of sequences to two known graphene-binding peptide sequences
	(GrBP5 and M6), our initial goal was not to determine the biological
	function responsible for graphene binding, but to downselect three
	neuropeptides for experimentation, after which we could apply
	statistical methods to determine the most likely method of binding in
	the case where the candidates are successfully selected binding
	peptides. As such, we followed Cosic's method to determine the cross
	spectrum of each neuropeptide's EIIP and biological mapping spectra, and
	generated a cross spectra with the EIIP and biological mapping spectra
	of GrBP5 and M6. In this cross spectrum, the signal to noise ratio may
	be considered a primitive measure of ``similarity'' in biological
	function between the two peptides, should the RRM be valid in and of
	itself for determining such a function through characteristic peaks,
	because the S/N measures the degree to which constructive interference
	occurs between the two spectra, indicating similar biological function
	(which, in our case, is the potential to bind to graphene).
	
	\paragraph{Cross entropy}\label{cross-entropy}
	
	We determined the cross entropy for both the pure amino acid sequences
	as well as the functional domain numerical sequences, which can be found
	in the appendix (and which will be generated in the main program).
	
	\paragraph{PAM30 Sequence similarity}\label{pam30-sequence-similarity}
	
	The PAM (point accepted mutation) matrices, initially developed by
	Margaret Dayhoff{[}\^{}5{]} are commonly used in measuring the
	similarity of two peptide or protein sequences. The PAM30 matrix,
	specifically, is a sequence alignment matrix that allows 30 point
	accepted per 100 amino acids. The PAM30 matrix is a ``shallow'' sequence
	alignment matrix, in that it is more appropriate in determining
	alignment for more ``similar'' sequences.
	
	\subsection{Results}\label{results}
	
	Fundamentally, our results were calculated on the basis of two
	well-established similarity matrices (the PAM30 substitution matrix and
	the BLOSUM substitution matrix) as well as two key-value lookups
	associating an amino acid with a scalar quantity representing its
	notable biological function (in the case of the statistical clustering
	key:value pairing) or its propensity for electron-ion interaction (in
	the case of EIIP). Since these latter two methods represent our most
	novel foray into characterizing peptide similarity, we first determined
	their correlation to one another. Since the biological function mapping
	is categorical, and the EIIP mapping is continuous, we utilizd the
	Kendall Tau correlation parameter (commonly used to compare categorical
	and continuous data) and found these two mappings to have a correlation
	(tau) of 0.1129 and corresponding p\_value of 0.5083, indicating very
	slight agreement.
	
	\paragraph{Candidate peptides}\label{candidate-peptides}
	
	The five similarity metrics generated for each neuropeptide served as a
	feature set which could be weighted and used to train a regression
	model, but without experimental data, the relative weights for each
	feature were chosen by {[}METHOD{]}.
	
	After forwarding the candidate peptides for experimentation, we utilized
	statistical clustering and signal processing methods to predict
	determinant sequence domains in graphene binding which could then be
	used with the experimental data once acquired.
	
	\paragraph{Experimental results}\label{experimental-results}
	
	After generating the list of {[}NUMBER{]} candidate peptides and
	receiving binding affinity metrics, we were then able to perform several
	statistical methods to ascertain the most influential factors in peptide
	graphene binding.
	
	First, we trained a simple linear regression model, using the gathered
	experimental results as training data, and {[}\ldots{}{]}
	
	\paragraph{Signal transduction}\label{signal-transduction}
	
	From results gathered from model training, statistical clustering, and
	experimental cross-validation, we were able to identify several sequence
	patterns and corresponding functions which may prove significant.
	{[}\ldots{}{]}
	
	\subsection{Discussion}\label{discussion}
	
	\paragraph{Computational results}\label{computational-results}
	
	Our program, using all aforementioned computational methods, generated
	two ``similarity tables'' for 934,235 {[}@TODO: revise this{]}
	neuropeptides: one for GrBP5, and one for the wild-type peptide M6. This
	generated a table of the schema (example data -- replace with top 10-20
	candidate peptides):
	
	\textbf{\emph{Table 1:}} \emph{Output schema of .csv similarity tables}
	
	\begin{longtable}[]{@{}lllllll@{}}
		\toprule
		\textbf{seq} & numseq & rrmsn & rrmcorr & pam30 & numentr &
		aaentr\tabularnewline
		\midrule
		\endhead
		IMVSTED & 1132335 & 73 & 88 & 23 & 45 & 55\tabularnewline
		GTTYUEI & 3224121 & 11 & 5 & 6 & 14 & 13\tabularnewline
		\bottomrule
	\end{longtable}
	
	Where the columns correspond as follows:
	
	\begin{itemize}
		\tightlist
		\item
		\textbf{seq}: Original amino acid sequence of the neuropeptide
		\item
		\textbf{numseq}: The original amino acid sequence converted to numbers
		corresponding to their biological function as determined by the
		biological function key-value table{[}\^{}6{]}
		\item
		\textbf{rrmsn}: The signal-to-noise ratio of the cross-signal of the
		EIIP frequency spectum calculated for a given neuropeptide and the
		sequence of comparison, normalized in the domain {[}0, 100{]}.
		\item
		\textbf{rrmcor}: The Pearson correlation coefficient between the EIIP
		frequency spectrum calculated for a given neuropeptide and the
		sequence of comparison, mapped to the domain {[}0, 100{]}
		\item
		\textbf{pam30}: The PAM30 similarity score (formula described in
		methods section); a measure of inverse distance using the PAM30
		matrix, mapped to {[}0, 100{]}
		\item
		\textbf{numentr}: The cross-entropy of the numerical sequence
		represencted by \texttt{numseq}
		\item
		\textbf{aaentr}: The cross-entropy of the amino acid sequence
	\end{itemize}
	
	From this data, we determined that {[}\ldots{}{]}
	
	\paragraph{Experimental results}\label{experimental-results-1}
	
	{[}to be completed after experimentaiton{]}
	
	\paragraph{Clustering and signal
			analysis}\label{clustering-and-signal-analysis}
	
	After gathering the experimental results, we were then able to apply
	supervised statistical learning techniques to the full set of
	neuropeptides, as well as signal processing techniques. We first
	{[}\ldots{}{]}
	
	\subsection{How to run}\label{how-to-run}
	
	\textbf{Prerequisites}: A computer running Windows/Mac OSX/Linux with
	\href{https://www.python.org/}{Python} (3+) installed, and the Python
	libraries \href{https://pandas.pydata.org/}{pandas},
	\href{https://numpy.org/}{NumPy}, and
	\href{https://scikit-learn.org/stable/}{scikit-learn}. For
	visualizations, the library \href{https://matplotlib.org/}{matplotlib}
	should be installed. If these libraries are not already installed,
	instructions for their installation will be listed below.

\end{document}
